% !TeX root = ../main.tex

\ustcsetup{
  keywords = {
    云计算, 微服务, 应用编排, 低代码, 弹性计算
  },
  keywords* = {
    Cloud computing, microservices, application orchestration, low code, elastic computing
  },
}

\begin{abstract}
  本文介绍了一种应用与服务编排工作流的云服务系统,系统的职责是为企业和个人提供应用上云的自动化编排能力。
  提供编排一个或多个云服务的能力,将现有的云服务作为一个组件,在系统中可以根据用户的意愿自由组合起来,根据模板或者完全自定义的方式来创建和配置现
  有的云资源和应用,以构建一个应用工作流,系统将根据用户构建的工作流提供的流程、资源描述信息,自动地进行解析,将其抽象为一个有向无环图作为执行引
  擎的输入对象,执行的输出就是工作流编排的各个任务执行所产生的输出。并且,提出低代码开发的方式,以可视化辅以JSON的混合开发模式进行应用服务的开发
  \cite{othe4},进一步降低用户门槛。

在积极地回访收集内部测试用户的有效建议后,了解到目前由于业务高速扩展,系统需要保证在数据量不断攀升的情况下,仍能较好地保证系统的可用性。
  因而,对系统的可用性、一致性提出了较高的要求,本文针对这些方面,提出了详尽的设计方案,对方案的可行性作出了探讨,
  提出了一种以当下最为灵活的微服务架构模式进行系统功能模块的划分的方式的设计方案,通过对系统瓶颈环节进行需求分析,提出可行的低成本方案,
  有针对性地吸取其它同类系统的经验,据此改进系统\cite{othe3},适应目前同类产品所欠缺的各项能力。包含降低用户门槛的应用模板部署能力、高效执行的并行化等能力。
  通过合理的模块划分,数据对象建模,数据库设计等工作。设计目的就是构建一个满足高性能、高可用、高可扩展性的分布式系统。



\end{abstract}
\begin{abstract*}
  This paper introduces a cloud service system for application and service orchestration workflow. The responsibility of the system is to provide enterprises and individuals with the ability of automatic orchestration on the application cloud.

  It provides the ability to arrange one or more cloud services. The existing cloud services are regarded as a component, which can be freely combined in the system according to the wishes of users, and can be created and configured according to templates or completely customized methods

  Some cloud resources and applications to build an application workflow. The system will automatically analyze the process and resource description information provided by the workflow built by the user, and abstract it into a directed acyclic graph as an execution guide

  The input object of the engine, and the output of execution is the output generated by the execution of each task arranged by the workflow. In addition, a low code development mode is proposed to develop application services with the hybrid development mode of visualization and JSON

  , further reduce the user threshold.

  After actively revisiting and collecting effective suggestions from internal test users, it is learned that due to the rapid expansion of business, the system needs to ensure that the availability of the system can be better guaranteed under the condition of increasing data volume.

  Therefore, higher requirements are put forward for the availability and consistency of the system. This paper puts forward a detailed design scheme for these aspects, and discusses the feasibility of the scheme,

  This paper puts forward a design scheme of dividing the system functional modules with the most flexible microservice architecture mode at present, and puts forward a feasible low-cost scheme through the demand analysis of the system bottleneck,

  Learn from the experience of other similar systems, and improve the system accordingly to adapt to various capabilities lacking in similar products at present. It includes application template deployment ability to reduce user threshold, parallelization of efficient execution, etc.

  Through reasonable module division, data object modeling, database design and so on. The design purpose is to build a distributed system with high performance, high availability and high scalability.

\end{abstract*}
