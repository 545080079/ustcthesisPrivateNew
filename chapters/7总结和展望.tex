%! Author = Administrator
%! Date = 2021/7/2

\chapter{总结和展望}
本文从实习过程中在公司所参与的应用与服务编排工作流ASW项目入手,详细描述了一种高并发、高可用系统的后台架构的设计方案。该应用目前已经发布
至公有云投入使用,聚合了整个公有云的产品提供的服务能力,紧随当前公有云服务的大一统趋势,以及低代码配置编程的探索,对微服务应用编排提出
一种崭新的方式来编排云上业务,旨在简化开发人员日常琐碎重复的工作,提高研发流程效率。

本应用按照软件工程所提出的理论,测试驱动开发的思想,遵循了先进的研发流程,从需求出发,先对整体需求进行分析设计,提出可测试的方案,只要以该标准
来编写代码,就能最大程度保障系统最终的完成度和实用性。这也让我将学校学习的理论知识第一次投入到了实际的工业生产环境当中,深化了对设计模式、软件架构、需求分析等
的理解,而不仅仅停留于表面的了解。

综上,本文从以下几个角度分析了系统的架构:
\begin{enumerate}
    \item 加入缓存层,提供强大的对并发能力的数据读写需求
    \item 基于预测执行,优化执行器的效率
    \item 更加合理的划分微服务模块,对应用进行模块化管理
    \item 加强了基础能力建设,构建自动化测试能力以及日志管理能力
\end{enumerate}
基于以上几点论述,本论文完成了对一个日均处理数据量在十亿量级的一个高可用系统的设计工作,并且将其转化为一款成熟的商业化产品,在敏捷迭代
的过程中,由于设计的可扩展性,不再具有明显的瓶颈短板,可以很好地适应当前环境。


\section{论文总结}


\section{问题和展望}

通过ASW自身的特点,输入数据的可预测性,提出了预测执行的方式来提升执行效率,这一点仍具有很大的改进空间。后期,当用户执行数据量达到一定程度时,
可以通过机器学习的方式,对输入数据进行深度处理,这样,更加复杂的输入输出类型也可以进行预测,预测执行成功率将大大提升。